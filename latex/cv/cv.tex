\documentclass[letterpaper,12pt,oneside]{article}
\usepackage[utf8]{inputenc}
\usepackage{setspace}
\usepackage{hyperref}
\usepackage{booktabs}% http://ctan.org/pkg/booktabs
\usepackage{longtable}
\usepackage{graphicx}
\usepackage{fancyhdr}
\usepackage{enumitem}
\usepackage[left=1in, right=1in, bottom=1in, top=1.25in]{geometry}
\usepackage{fontspec} % custom fonts
\usepackage{titlesec}
\pagenumbering{arabic}

\setmainfont[
    Path=/usr/share/fonts/TTF/,
    BoldFont=Roboto-Bold.ttf,
    ItalicFont=Roboto-Italic.ttf,
    BoldItalicFont=Roboto-BoldItalic.ttf
]{Roboto-Light.ttf}

\newfontfamily\regular[
    Path = /usr/share/fonts/TTF/
]{Roboto-Regular.ttf}

\newfontfamily\light[
    Path = /usr/share/fonts/TTF/
]{Roboto-Light.ttf}

\newfontfamily\thin[
    Path = /usr/share/fonts/TTF/
]{Roboto-Thin.ttf}

\newfontfamily\regularheader[
    Path = /usr/share/fonts/TTF/
]{Roboto-Regular.ttf}

\newfontfamily\thinheader[
    Path = /usr/share/fonts/TTF/
]{Roboto-Thin.ttf}

\newfontfamily\mediumheader[
    Path = /usr/share/fonts/TTF/
]{Roboto-Medium.ttf}


\pagestyle{fancy}
\fancyhf{}
\chead{\LARGE{\textbf{Peijun Ma}} \\ \small \href{mailto:peijun.ma@pm.me}{peijun.ma@pm.me} | (647) 809-5248 | \href{https://www.linkedin.com/in/peijun-ma/}{https://www.linkedin.com/in/peijun-ma/} | \href{https://peijun.dev}{https://peijun.dev}}

\newcommand{\smallurl}[1]{\footnotesize{\url{#1}}\normalsize}

\titlespacing*{\section}{0pt}{1.0ex}{0.5ex}
\titlespacing*{\subsection}{0pt}{0.7ex}{0.3ex}
\titlespacing*{\subsubsection}{0pt}{0.5ex}{0.2ex}

\begin{document}

\section*{Skills}
\subsection*{Technology}
\subsubsection*{Expert}
I am able to onboard new team members with these technologies:\\
Go, Scala, Python, Docker, Terraform, Google Cloud Platform, Linux, CircleCI
\subsubsection*{Proficient}
I am able to be immediately productive with these technologies:\\
Clojure, AWS, Kubernetes, TypeScript, Angular, Java, Jenkins
\subsubsection*{Familiar}
I am able to pick up these technologies relativity quickly:\\
Haskell, Rust, Vue.js, C\#, AWS, C, React, Next.js
\subsection*{Other}
\begin{itemize}
      \setlength\itemsep{0em}
      \item Designing and developing distributed systems
      \item Architecting software systems using domain-driven and service oriented principles
      \item Developing continuous deployment workflows to Kubernetes in a production environment
      \item Administrating Unix-like operating systems, have knowledge of the OSI network model
      \item Writing technical documentation and producing infrastructure diagrams
      \item Solid understanding of data structures and algorithms
      \item Knowledgeable in agile (XP, Kanban, and Scrum)
\end{itemize}

\section*{Experience}
\subsection*{Thumbtack}
Site Reliability Engineer | March 2023 -- Present
\begin{itemize}
    \setlength\itemsep{0em}
    \item
\end{itemize}

\subsection*{CircleCI}
Software Engineer | June 2021 -- December 2022
\begin{itemize}
    \setlength\itemsep{0em}
    \item Improved API coherency and tolerance of external faults by designing and implementing an \textbf{anti-corruption layer}
    \item Reduced observability costs by \textbf{25\%} while preserving ability to diagnose issues by auditing and consolidating usage of spans
    \item Increased customer engagement for self-hosted runner product by designing runner APIs and UIs
    \item Improved reliability for \textbf{80\%} of customer jobs by designing and implementing the deprecation of legacy services
    \item Reduced storage costs by \textbf{50\%} by designing and implementing customer configurable storage retentions
    \item Improved software delivery confidence by designing comprehensive \textbf{end-to-end testing} framework
\end{itemize}

\subsection*{Garner Distributed Workflow}
Software Engineer | November 2018 -- April 2021
\begin{itemize}
      \setlength\itemsep{0em}
      \item Improved software delivery speed and consistency by implementing the \textbf{GitOps} process using \textbf{ArgoCD}
      \item Increased horizontal scalability of the production system by 30\% by migrating the system to \textbf{Kubernetes}
      \item Granted the ability to improve the production infrastructure to the dev team by developing infrastructure as code project using \textbf{Terraform}
      \item Optimized the performance of graph DB queries by 2000x by reducing the runtime complexity of result parsing from $\mathcal{O}(n^3)$ to $\mathcal{O}(n\log{}n)$
      \item Improved the maintainability of the codebase by championing \textbf{functional programming} with weekly study sessions
      \item Reduced CI costs by \textbf{60\%} by migrating from Jenkins to Google Cloud Build
\end{itemize}

\section*{Projects}

\subsection*{Personal website \hfill \normalfont{Open Source project}}
\smallurl{https://gitlab.otonokizaka.moe/Umi/peijun.dev}
\begin{itemize}
      \setlength\itemsep{0em}
      \item Built a personal website using \textbf{Vue.js}
      \item Published the website as a \textbf{Docker} image using GitLab CI
      \item Deployed the website automatically using \textbf{GitLab CI}
\end{itemize}

\subsection*{Cloudflare DDNS \hfill \normalfont{Open Source project}}
\smallurl{https://github.com/MaT1g3R/cloudflare-ddns}
\begin{itemize}
      \setlength\itemsep{0em}
      \item Purposed Cloudflare's DNS service as dynamic DNS using \textbf{Terraform}
\end{itemize}

\subsection*{Office hour scheduler \hfill \normalfont{Open Source project}}
\smallurl{https://github.com/office-hour-scheduler/ohs}
\begin{itemize}
      \setlength\itemsep{0em}
      \item Collaborated with several other developers to build a webapp for scheduling office hours with professors
      \item Designed the backend using \textbf{domain-driven} design principles
      \item Implemented the backend API using \textbf{GraphQL}
      \item Built the artifact as a Docker image in CI
\end{itemize}

\subsection*{Option (Python Library) \hfill \normalfont{Open Source project}}
\smallurl{https://github.com/MaT1g3R/option}
\begin{itemize}
      \setlength\itemsep{0em}
      \item Implemented a library to bring Rust-like Optional types to Python
      \item Integrated automated testing and deployment using Travis CI and codecov
      \item Published the library automatically to PyPi using the CI pipeline
\end{itemize}

\subsection*{Chat bot \hfill \normalfont{Open Source project}}
\smallurl{https://github.com/MaT1g3R/YasenBaka}
\begin{itemize}
      \setlength\itemsep{0em}
      \item Created a chat bot serving 1000+ chat rooms at its peak
      \item Explored asynchronous programming using the \textbf{asyncio} library
      \item Deployed the application using Docker to \textbf{AWS}
\end{itemize}

\subsection*{Music player \hfill \normalfont{Open Source project}}
\smallurl{https://github.com/MaT1g3R/musicview}
\begin{itemize}
      \setlength\itemsep{0em}
      \item Designed a music player that discovers the least played songs for the user
      \item Created a simple UI using ncurses
      \item Synchronized between different threads using locks and condition variables
\end{itemize}

\subsection*{ext2 file system \hfill \normalfont{Course project, University of Toronto}}
\begin{itemize}
      \setlength\itemsep{0em}
      \item Implemented a ext2 file system from scratch using C
      \item Debugged the program using GDB, valgrind, and CLion
      \item Experimented with the CMake build system
\end{itemize}

\section*{Education}
\subsection*{University of Toronto \hfill \normalfont{September 2016 -- June 2021}}
Honours Bachelor of Science, \textbf{Computer Science Specialist}
\end{document}
