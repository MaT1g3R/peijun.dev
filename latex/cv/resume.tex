\documentclass[letterpaper,12pt,oneside]{article}
\usepackage[utf8]{inputenc}
\usepackage{setspace}
\usepackage{hyperref}
\usepackage{booktabs}% http://ctan.org/pkg/booktabs
\usepackage{longtable}
\usepackage{graphicx}
\usepackage{fancyhdr}
\usepackage{enumitem}
\usepackage[left=1in, right=1in, bottom=1in, top=1.25in]{geometry}
\usepackage{fontspec} % custom fonts
\usepackage{titlesec}
\pagenumbering{arabic}

\setmainfont[
    Path=/usr/share/fonts/TTF/,
    BoldFont=Roboto-Bold.ttf,
    ItalicFont=Roboto-Italic.ttf,
    BoldItalicFont=Roboto-BoldItalic.ttf
]{Roboto-Light.ttf}

\newfontfamily\regular[
    Path = /usr/share/fonts/TTF/
]{Roboto-Regular.ttf}

\newfontfamily\light[
    Path = /usr/share/fonts/TTF/
]{Roboto-Light.ttf}

\newfontfamily\thin[
    Path = /usr/share/fonts/TTF/
]{Roboto-Thin.ttf}

\newfontfamily\regularheader[
    Path = /usr/share/fonts/TTF/
]{Roboto-Regular.ttf}

\newfontfamily\thinheader[
    Path = /usr/share/fonts/TTF/
]{Roboto-Thin.ttf}

\newfontfamily\mediumheader[
    Path = /usr/share/fonts/TTF/
]{Roboto-Medium.ttf}


\pagestyle{fancy}
\fancyhf{}
\chead{\LARGE{\textbf{Peijun Ma}} \\ \small \href{mailto:peijun.ma@pm.me}{peijun.ma@pm.me} | (902)-719-8619 | \href{https://www.linkedin.com/in/peijun-ma/}{https://www.linkedin.com/in/peijun-ma/} | \href{https://peijun.dev}{https://peijun.dev}}

\newcommand{\smallurl}[1]{\footnotesize{\url{#1}}\normalsize}

\titlespacing*{\section}{0pt}{1.0ex}{0.5ex}
\titlespacing*{\subsection}{0pt}{0.7ex}{0.3ex}
\titlespacing*{\subsubsection}{0pt}{0ex}{0ex}

\begin{document}

\section*{Experience}
\subsection*{Garner Distributed Workflow \hfill \normalfont{November 2018 -- Present}}
Software engineer
\begin{itemize}
    \setlength\itemsep{0em}
    \item Improved software delivery speed and consistency by implementing the \textbf{GitOps} process using \textbf{ArgoCD}
    \item Increased horizontal scalability of the production system by 30\% by migrating the system to \textbf{Kubernetes}
    \item Granted the ability to improve the production infrastructure to the dev team by developing infrastructure as code project using \textbf{Terraform}
    \item Optimized the performance of graph DB queries by 2000x by reducing the runtime complexity of result parsing from $\mathcal{O}(n^3)$ to $\mathcal{O}(n\log{}n)$
    \item Improved the maintainability of the codebase by championing \textbf{functional programming} with weekly study sessions
\end{itemize}

\section*{Skills}
\textbf{Proficient}:
Scala, Python, Docker, Terraform, Google Cloud Platform, Linux,
Kubernetes, TypeScript, Angular, Java, Jenkins \\
\textbf{Familiar}:
Haskell, Rust, Vus.js, C\#, AWS, C

\section*{Projects}

\subsection*{Office hour scheduler \hfill \normalfont{Open Source project}}
\smallurl{https://github.com/office-hour-scheduler/ohs}
\begin{itemize}
    \setlength\itemsep{0em}
    \item Collaborated with several other developers to build a webapp for scheduling office hours with professors
    \item Designed the backend using \textbf{domain-driven} design principles
    \item Implemented the backend API using \textbf{GraphQL}
    \item Built the artifact as a Docker image in CI
\end{itemize}

\subsection*{Chat bot \hfill \normalfont{Open Source project}}
\smallurl{https://github.com/MaT1g3R/YasenBaka}
\begin{itemize}
    \setlength\itemsep{0em}
    \item Created a chat bot serving 1000+ chat rooms at its peak
    \item Explored asynchronous programming using the \textbf{asyncio} library
    \item Deployed the application using Docker to \textbf{AWS}
\end{itemize}

\section*{Education}
\subsection*{University of Toronto \hfill \normalfont{September 2016 -- April 2021}}
Bachelor of Science, \textbf{Computer Science Specialist}
\end{document}
